\documentclass[10pt,pdf,hyperref={unicode}]{beamer}

\usepackage{lmodern}

\usepackage[T2A]{fontenc}
\usepackage[utf8]{inputenc}

\usepackage{ragged2e}
\newcommand{\jj}{\righthyphenmin=20 \justifying}

\setbeamertemplate{navigation symbols}{}

\usetheme{CambridgeUS}

\usecolortheme{seahorse}

\title{Игра "Слова из слова"}   
\author{Студентка 3 курса\\Специальности «Компьютерная безопасность»\\Тронина София} 
\date{2022 год} 

\begin{document}
	
	\begin{frame}
		\titlepage
	\end{frame} 
	
   	\begin{frame}
		\frametitle{Суть игры}
		\jj
		Рандомно выбирается слово, из которого можно составить другие слова. \\
		~\\Первым ходит игрок. После того как он записал свое слово, компьютер составляет свое. Каждая буква может встречаться в слове один раз и слова не могут повторяться. Минимальная длина слова - 3 буквы. За каждую использованную букву дается 1 балл. \\
		~\\Игра заканчивается, если все возможные слова были введены или игрок нажал на кнопку "Закончить игру". \\
		~\\Игрок выигрывает, если у него на счету больше баллов, чем у компьютера.
	\end{frame}

	\begin{frame}
		\frametitle{Описание}
		\jj
		После импорта всех нужных библиотек (tkinter, random, messagebox). 
		
		Необходимо выбрать основное слово.
		\begin{figure}[h]
			\includegraphics[width=0.7\linewidth]{1.png}
			\label{fig:mpr}
		\end{figure}
		\jj
		Также необходимо ввести некоторые дополнительные пометки:
		\begin{figure}[h]
			\includegraphics[width=0.8\linewidth]{2.png}
			\includegraphics[width=0.8\linewidth]{3.png}
		\end{figure}
	\end{frame}

	\begin{frame}
		\jj
		Затем создадим класс для кнопок, где пропишем кнопки для ввода букв, и также дополнительные кнопки "Записать" и "Очистить". 
		\begin{figure}[h]
			\includegraphics[width=0.7\linewidth]{4.png}
		\end{figure}
		\jj
		У кнопок-букв присутствует две команды. Первая отвечает за изменение статуса кнопки, а вторая - вывод буквы на экран:
		\begin{figure}[h]
			\includegraphics[width=0.9\linewidth]{5.png}
		\end{figure}
	\end{frame}
	
	\begin{frame}
		\jj
		Кнопка "Записать" имеет одну команду, но она включает в себя несколько подпунктов: запись слова игрока и компьютера, подсчет баллов, появления информационных окон в случаях, если введенное слово уже присутствует на экране или все возможные слова были введены. 
		 \begin{figure}[h]
		 \includegraphics[width=0.8\linewidth]{6.png}
		 \includegraphics[width=0.9\linewidth]{7.png}
		 \end{figure}
	\end{frame}
	
	\begin{frame}
		\begin{figure}[h]
		\includegraphics[width=0.8\linewidth]{8.png}
		\end{figure}
		\jj
		Количество возможных слов считается с помощью дополнительной функции:
		\begin{figure}[h]
			\includegraphics[width=0.8\linewidth]{9.png}
		\end{figure}
	\end{frame}
	
	\begin{frame}
		\jj
		Кнопка "Очистить" также имеет одну команду: она стирает написанное слово полностью:
		\begin{figure}[h]
			\includegraphics[width=0.7\linewidth]{10.png}
		\end{figure}
	\end{frame}
	
	\begin{frame}
		\frametitle{Формирование окна}
		\jj
		Теперь необходимо приступить к оформлению самого окна игры:
		\begin{figure}[h]
			\includegraphics[width=0.7\linewidth]{11.png}
		\end{figure}
	\end{frame}

	\begin{frame}
		\jj
		С помощью цикла формируем буквы-кнопки:
		\begin{figure}[h]
			\includegraphics[width=0.7\linewidth]{12.png}
		\end{figure}
		\jj
		Затем необходима еще одна кнопка: "Закончить игру". Ее можно нажать в любой момент игры.
		\begin{figure}[h]
			\includegraphics[width=1.0\linewidth]{13.png}
		\end{figure}
		\jj
		Она имеет две команды: вывод информационного окна с информацией о том, кто выиграл, и закрытие основного окна:
		\begin{figure}[h]
			\includegraphics[width=1.0\linewidth]{14.png}
		\end{figure}
	\end{frame}

	\begin{frame}
		\frametitle{Вид игры}
		\begin{figure}[h]
			\includegraphics[width=1.0\linewidth]{15.png}
		\end{figure}
	\end{frame}

	\begin{frame}
		\frametitle{Стратегия игры}
		\jj
		Так как в свой ход компьютер записывает слова, которые ранее еще не использовались, то в начале лучше составлять слова как можно длиннее, а затем переходить к более коротким. Таким образом, можно набрать баллов больше, чем у компьютера.
	\end{frame}
\end{document}